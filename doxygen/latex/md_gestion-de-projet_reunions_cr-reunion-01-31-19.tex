{\itshape {\bfseries Deadline diagramme de Gantt 22 fev}}
\begin{DoxyItemize}
\item Possibilité de modifier pendant le projet
\end{DoxyItemize}

Important de réaliser les tests unitaires en continue Objectif \+: Couverture de 100\% sur le Modèle

Après le diagramme de Gantt\+:
\begin{DoxyItemize}
\item écriture des .h pour décrire le programme
\item éventuellement U\+ML
\item création du git
\item système d\textquotesingle{}intégration continue sur git
\item un dossier sur chaque partie M, V, C
\item sur la branche master \+: uniquement les versions
\end{DoxyItemize}

{\itshape {\bfseries deadline git 29/05 à 14h}}

\subsubsection*{Réunions}


\begin{DoxyItemize}
\item Chaque réunion donne lieu à un rapport
\end{DoxyItemize}

\subsubsection*{Documentation}


\begin{DoxyItemize}
\item Documentation des prototypes
\item Doxygen (uniquement le modèle est demandé en couverture 100\% doxygen mais c\textquotesingle{}est mieux de la faire pour tout)
\item Réaliser des tests Valgrind et Tests unitaires (C\+Mocka, gcov, ...) pour chaque fonction codée
\item (Optionnel) Utilisation de C\+Make pour générer les makefiles
\end{DoxyItemize}

\subsubsection*{Rapport final}


\begin{DoxyItemize}
\item 5 pages max \+:
\begin{DoxyItemize}
\item détails des objectifs
\item détails des choix techniques
\item méthodologie (organisation du projet)
\item parles des modifications apportées au Gantt
\item résultats du projet, rapport au cahier des charges
\end{DoxyItemize}
\item important d\textquotesingle{}écrire le rapport au fur et a mesure
\item {\itshape {\bfseries Deadline rapport 23/05/19}}
\item possibilité de faire une démonstration en live (mieux vaut prévoir une vidéo au cas où)
\item (optionnel) vidéo de présentation 
\end{DoxyItemize}
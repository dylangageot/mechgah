\documentclass[11pt]{book}

\usepackage{packageINSA}

\begin{document}
\pagestyle{plain}
\fontfamily{cmss}
\selectfont

% LES DIFFERENTS CHAMPS DE LA COUVERTURE
\sautverticalnegatif{1.4}
% Valeur en cm du saut vertical négatif (ie. vers le haut)
% Modifier cette valeur si le prénom et le nom de l'auteur ne sont pas positionnés au bon endroit par rapport à la ligne "Projet de Fin d'Etudes"
% Si le prénom et le nom tiennent sur la ligne, laisser valeur = 1.4
\specialite{EII}  % Nom de la spécialité
\anneeuniversitaire{2018 - 2019}  % Année universitaire
\titre{Emulateur de la Nintendo Entertainment System (NES)}  % sujet du PFE
\correspondantINSA{Alexandre}{TISSIER}  % Prénom et nom du correspondant INSA
\datesout{27/05/2019}  % Date de la soutenance du PFE

% RESUMES EN FRANÇAIS ET EN ANGLAIS
\resumefrancais{\textsf{Remplacer ce texte par le résumé en français}}
\resumeanglais{\textsf{Change this text by the abstract in english}}
% Les résumés en français et en anglais sont positionnés cote à cote
% Une hauteur de cadre a été définie pour que les zones réservées aux résumés tiennent sur une seule page : ne pas la modifier

\makeTitlePage % Page de titre / première page

\clearpage

\chapter{}

Coucou

\section{Introduction}

Introduction

\makeAbstractPage % la page avec les résumés (français et anglais)

\end{document}
